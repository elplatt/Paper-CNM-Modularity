\documentclass{article}

\newcommand{\beq}{\begin{eqnarray}}
\newcommand{\eeq}{\end{eqnarray}}

\begin{document}

The Clauset-Newman-Moore (CNM) community detection algorithm \cite{clauset_finding_2004}
greedily maximizes the {\em modularity} $Q$ by repeatedly merging smaller communities.
The orginal presentation is given for an unweighted, undirected network with no self-loops.
This paper derives analogoous equations for weighted networks with self-loops,
both directed and undirected.

Throughout this paper, $A_{vw}$ refers to an entry in the adjacency matrix of a network.
Each end of an {\em edge} connects to a {\em vertex},
which can be the same vertex for both edges in the case of a self-loop.
A {\em stub} is one half of an edge, and is associated with single vertex.
Each edge has a {\em weight}, with $m$ refering to the total edge weight.
In an {\em undirected} network, edges are symmetric with respect to vertices.
In a {\em directed} network, the edge is outgoing from the {\em source} vertex
and incident upon the {\em sink} vertex.
Respectively, the degree $k_v$, in-degree $k_v^{(in)}$, and out-degree $k_v^{(out)}$ refer to the
sum of adjacent edge weights, sum of incident edges, and sum of outgoing edges.
The term $\delta_{ij}$ is equal to 1 if $i = j$ and 0 otherwise.

\section{Undirected}
In an undirected network, modularity is defined as:
\beq
\label{eq:modularity} Q &=& \frac{1}{2m} \sum_{v,w} \left[ A_{vw} (1 + \delta_{vw}) - \frac{k_v k_w}{2m} \right]\delta_{c_v c_w} \\
m &=& \frac{1}{2} \sum_{v,w} A_{vw} (1 + \delta_{vw}),
\eeq
where $m$ is the total number of edges is the network.
Equation \ref{eq:modularity} differs from \cite{clauset_finding_2004} by the inclusion of a $\delta_{vw}$ term.
This term is necessary when self-loops are present,
because unlike off-diagonal entries, diagonal entries represent two stubs.

\begin{thebibliography}{1}
\bibitem{clauset_finding_2004} 
Aaron Clauset, M. E. J. Newman, and Christopher Moore.
\textit{Finding community structure in very large networks}. 
Physical review E 70(6):066111, 2004.
\end{thebibliography}

\end{document}
